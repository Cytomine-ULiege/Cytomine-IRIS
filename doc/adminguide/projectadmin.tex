%%%%%%%%%%%%%%%%%%%%%%%%%%%%%%%%%%%%%%%%%%%%%%%%%%%%%%%
% IRIS-specific project administration
%%%%%%%%%%%%%%%%%%%%%%%%%%%%%%%%%%%%%%%%%%%%%%%%%%%%%%%
\part{IRIS Administration}
\label{part:irisadmin}
{
\hypersetup{linkcolor=black}
\parttoc
}

\chapter{User Management}
In addition to the Cytomine user management, the IRIS module requires custom assignment of roles to users. 
This allows for custom widget visibility and other functionality in the IRIS interface as well as access to backend and database. 

\def\irisadmin{\texttt{IRISAdmin}}
\def\pjadmin{\texttt{ProjectAdmin}}
\def\pjcoord{\texttt{ProjectCoordinator}}
\section{User Roles}
\subsection{IRISAdmin}
The \irisadmin\ role gives users root access to all configurations and to the IRIS database, activity logs and other backend functionality within the Grails application.  
\warnbox{Caution}{Root access should be granted to just one person only!}

\subsection{IRISProjectAdmin}
A \pjadmin\ is able to manipulate other IRIS users' role assignments. 
This role also incorporates all functions and rights granted to \pjcoord .

The rights granted to a \pjadmin\ are as follows:
\begin{itemize}
\item User role manipulation for the particular project, this role has been assigned to
\begin{itemize}
\item declaring other users as \pjadmin
\item declaring other users as \pjcoord
\end{itemize}
\item All rights granted to \pjcoord
\end{itemize}


\subsection{IRISProjectCoordinator}
A \pjcoord\ can view all statistics of the projects he is assigned to. 
Users are just able to view their particular projects, they have been assigned to in the Cytomine core application.
Thus, they can only be a \pjcoord\ in these projects. 
In a particular project, multiple persons can have the role \pjcoord . 

The rights granted to a \pjcoord\ are as follows:
\begin{itemize}
\item Locking/Unlocking the project and/or particular images in the project to particular users
\item Interobserver statistics: view and export
\end{itemize}



\section{Assigning Project Roles}





%\begin{table}
%\caption{\label{tab:name}}
%\begin{tabular}{c}
%
%\end{tabular}
%\end{table}







