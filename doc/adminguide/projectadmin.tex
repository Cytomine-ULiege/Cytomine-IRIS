%%%%%%%%%%%%%%%%%%%%%%%%%%%%%%%%%%%%%%%%%%%%%%%%%%%%%%%
% IRIS-specific project administration
%%%%%%%%%%%%%%%%%%%%%%%%%%%%%%%%%%%%%%%%%%%%%%%%%%%%%%%
\part{IRIS Administration}
\label{part:irisadmin}
{
\hypersetup{linkcolor=black}
\parttoc
}

%%%%%%%%%%%%%%%%%%%%%%%%%%%%%%%%%%%%%%%%%%%%%%%%%%%%%%%%%%%%%%%%%
\chapter{User Management}
In addition to the Cytomine user management, the IRIS module requires custom assignment of roles to users. 
This allows for custom widget visibility and other functionality in the IRIS interface as well as access to backend and database. 

\def\irisadmin{\texttt{IRISAdmin}}
\def\pjcoord{\texttt{ProjectCoordinator}}
\section{User Roles}
\subsection{IRISAdmin}
The \irisadmin\ role gives users root access to all configurations and to the IRIS database, activity logs and other backend functionality within the Grails application.  
\warnbox{Caution}{Root access should be granted to just one person only!}

\noindent 

\paragraph{Default Configuration.} 
The default \irisadmin\ has the following credentials after a clean install of IRIS:
\begin{itemize}
\item Username: \texttt{admin}
\item Password: \texttt{admin}
\end{itemize}


\subsection{Project Coordinator}
A \pjcoord\ is basically the managing user of an IRIS project. 
In addition to having access to the entire front-end and participating in the study, the rights granted to the project coordinator list as follows:
\begin{itemize}
\item manipulation of project access settings for \emph{other} users
\item manipulation of image access settings for \emph{all} users
\item declaring other users as \pjcoord
\item access to the project statistics of blinded projects\todo{what about non-blinded projects?}
\item access to the project synchronization interface
\end{itemize}

A \pjcoord\ can view all statistics of the projects he is assigned to. 
Users are just able to view their particular projects, they have been assigned to in the Cytomine core application.
Thus, they can only be a \pjcoord\ in these projects. 
In a particular project, multiple persons can have the role \pjcoord . 

%\begin{table}
%\caption{\label{tab:name}}
%\begin{tabular}{c}
%
%\end{tabular}
%\end{table}



%%%%%%%%%%%%%%%%%%%%%%%%%%%%%%%%%%%%%%%%%%%%%%%%%%%%%%%%%%%%%%%%%
\chapter{Project Management}

\section{General Settings}
\subsection{Cytomine Host}
Any project is created on the Cytomine host, where the project owner configures it properly and assigns the project users. 
Each user is required to have a valid Cytomine account. 

\subsection{Connecting a Project to IRIS}
Once a project has been created on Cytomine and is about to be used in an IRIS instance, the project coordinator has to sync the project on the IRIS host. 


 

\section{Interobserver Studies}
A proper project configuration is essential to conduct an interobserver study. 

There are a few rules to follow in order to make it work properly on IRIS, which we will elaborate on in the following sections. 

\subsection{Configuration of Blinded Projects}
This section covers the project configuration that needs to be done on the Cytomine host. 

\subsubsection{Cytomine Host}

\subsubsection{IRIS Host}


\section{Configuration of Non-Blinded Projects}
Although the IRIS labeling interface has mainly been developed for blinded labeling of the annotations, it is possible to edit a regular project on an IRIS host as well. 




%%%%%%%%%%%%%%%%%%%%%%%%%%%%%%%%%%%%%%%%%%%%%%%%%%%%%%%%%%%%%%%%%
\chapter{Synchronization}
The entire data model regarding annotations and is hosted on the Cytomine core and IRIS just caches the labeling progress for each user per each image, particular synchronization is required to optimize the data load on an IRIS server. 

\section{Labeling Progress Synchronization}
Manually, the particular \pjcoord\ can trigger the labeling progress synchronization for this project and all users in it by using the particular service endpoint. 
Synchronization can be triggered for each project separately using the corresponding tab in the project settings. 

\tobedone

\paragraph{Auto-Synchronization.}
The IRIS host automatically synchronizes the labeling progress for each user in each image that is enabled for that very same user. 
Per default, an entire auto-synchronization of all available users in their projects is done at startup of the instance, and once per day at 03:00 AM, such that daily business is not interrupted. 

\warnbox{Inconsistencies}{The auto-synchronization may lead to inconsistencies when the user is new to an IRIS host. The \pjcoord\ has to trigger the synchronization manually then to reflect the current labeling status for this user.}




